\section{History}
The story of JavaScript begins in 1995 when Netscape owned over 90\% browser market share after taking over from Mozaic. Netscape hired Brandon Eich to write Scheme in the browser. Scheme is a Lisp family language that was famous in the 90s\cite{findler_c}. After attempting to do a deal with Sun systems, it was decided that Brandon creates a language similar to Java but different. on top of being similar to Java, JavaScript was suppose to tolerate minor errors, have simplified event handling and make it easy to copy and paste code snippets.
\paragraph{}
The new language was called to be called Mocha which was later changed to LiveScript. By the end of 1995, Netscape and Sun systems had struck a deal and LiveScript was changed to JavaScript\cite{JS}. This was mainly done for marketing purposes, not meant to imply that the new language is a derivative of Java\cite{JS}. 
\paragraph{}
These were the days when Java was the dominating and most promising programming language. As a result the new language for the web was suppose to look like Java \cite{JS}. As a result JavaScript adopted a syntax similar to that of Java. It also inherited date objects that were similar to Java together with the Java y2k date bugs\cite{Eich:2005:JTY:1090189.1086382}. However the programming paradigm is not the same. This similarity in syntax can be a turn-off for most new developers trying out JavaScript. As the two languages differ especially when it comes to their object oriented programming paradigm.
\paragraph{}
JavaScript was meant to be an 'Object-based/ object oriented scripting' language \cite{Eich:2005:JTY:1090189.1086382}. For this Eich looked into an object oriented language called Self. Self's programming paradigm employs the concepts of prototypes \cite{Ungar:1987:SPS:38807.38828}. As result JavaScript adopted the use of prototypes in objects and prototypical inheritance which we see today in JavaScript. This means that all functions in JavaScript have a prototype field \cite{Eich:2005:JTY:1090189.1086382}. This allows all functions in JavaScript to be able to construct objects. Unlike most languages that use access modifiers for information hiding, JavaScript employs closures to achieve this.
\paragraph{}
Another paradigm which was discovered later by most developers in JavaScript was its functional programming capabilities\cite{JS}. This nature JavaScript had gotten in from Scheme, a Lisp family functional programming language. It meant that functions in JavaScript are to be treated as first  class citizens. The functional programming paradigm in JavaScript has been persisted  ever since and its still utilised today\cite{JS}.
\paragraph{}
The feature of tolerating minor errors made JavaScript to be permissive in most cases. For example a function can be invoked with too few or too many arguments. Types also convert freely e.g '1.0'==1 and '1'==1 both evaluate to true\cite{Eich:2005:JTY:1090189.1086382}. This allows bugs to hide in most programming projects.