\section{Variants of JavaScript}
The dramatic history of JavaScript resulted in an erroneous language that is non-conforming when compared to most of its peers. A lot of developers experience problems when delving deep into JavaScript due to its unusual features. Some of these features include dynamic typing, tolerance to errors and asynchronous call back functions. These problems lead to unnecessary bugs and performance issues on websites.
\paragraph{}
In the quest to make developing JavaScript applications easier and efficient, seasoned developers began to develop JavaScript frameworks. Some developed languages that compile to JavaScript that have a better typing system. On the other hand, others like jQuery were created to simplify and speed up DOM manipulation as vanilla JavaScript is deemed long and verbose. One could also argue and say that developers created frameworks to obey the principle of "Don't Repeat Yourself"(DRY) since most of the programming done in the web is repetitive. 
\paragraph{}
DOM manipulation is the most recurring task when using JavaScript. In order to make it simple and easy, jQuery was created. For the past ten years, jQuery has been the most used JavaScript framework. Its advantage is that you write few lines of code and achieve more. It therefore speeds up development and lowers development costs in the long run.
\paragraph{}
Being used for DOM manipulation and browser based applications, JavaScript was always thought of as a browser based language. However, the coming of Node.js in 2009 changed everything. Node.js is a JavaScript run-time environment that executes JavaScript code outside of a browser. Node.js allowed developers to developer server side applications using JavaScript. This made JavaScript even more popular in the software industry. It lead to the development of a package system for JavaScript and gave it the ability to manipulate binary files.Its non-blocking event loop was very attractive to most developers and companies. Node.js was built on top of the google V8 engine, which is a JavaScript engine for the Google chrome browser. V8 was chosen over Mozilla firefox SpiderMonkey and IE Chakra due to its speed of execution. Node.js excelled in easing the process of writing web apis and real-time applications compared to other famous web applications.
\paragraph{}
Even though developers were excited about the new developments in JavaScript especially the coming of Node.js, some were still not happy about the dynamic type system in JavaScript. This led to the creation of TypeScript in 2012 by Microsoft. TypeScript is a strict syntactical superset of JavaScript, and adds optional static typing to the language. TypeScript allows developers to write code that has less bugs compared to JavaScript. It is also less error tolerant and types don't just change.
\paragraph{}
Another language that has come up which is similar TypeScript is Dart. Dart includes both optional and static typing. It is mainly used for developing client side applications both for the web and mobile. Dart is still fairly new and so its advantages haven't been explored much.
\paragraph{}
Other variants include CoffeeScript, React, Angular and Vue. CoffeeScript's main goal is to expose the "good parts" of JavaScript. It is basically a little language that compiles to JavaScript \cite{coffee}. CoffeeScript allows developers to write few lines of code to produces code that produces that performs the same or better than code written JavaScript.
\paragraph{}
React, Angular and Vue are the most trending JavaScript frameworks today. These allow developers to develop websites in components. This is great because it makes the code modular and easier to maintain. They also make state management for complex applications easy \cite{react}. These component based libraries leverage a build tool called webpack. Webpack is a frontend tool used to bundle static assets in web development \cite{webpack}.