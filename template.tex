\documentclass[plain]{sigplanconf}
\usepackage{balance} % For balanced columns on the last page
\usepackage{amsmath}
\usepackage[T1]{fontenc}
\usepackage{lmodern}
\usepackage{graphicx}
\usepackage{amssymb}
\usepackage{tikz}
\usepackage{array}
\usepackage{longtable}
\usepackage{subcaption}
\usepackage[
bookmarksopen,
bookmarksdepth=2,
breaklinks=true
]{hyperref}
\usepackage{natbib}
\setcitestyle{square,sort,comma,numbers}
\makeatletter
\def\BState{\State\hskip-\ALG@thistlm}
\makeatother

\makeatletter
\def\@copyrightspace{\relax}
\makeatother
\begin{document}
	\title{Community Networks QoS Monitoring System}

	\authorinfo{Clayton Sibanda}
	{Department of Computer Science\linebreak University of Cape Town\linebreak South Africa}
	{}
	\maketitle

	\begin{abstract}
	\paragraph{}
	Characterising the internet through measurements has become very important to both end users and network
	managers. Most end users want to know what is causing their network based applications to slow down or take too long to respond. On the other hand network managers want to troubleshoot and detect faulty nodes in the network. The rise of cyber attacks and abuse of networks also makes it of paramount importance for network managers to continuously keep track of what is happening in their network.
	\paragraph{}
	In this paper we presented the design and creation of a visualiser for a quality of service monitoring platform for community networks. Such a visualiser is important in that it can help in monitoring network activity and identifying anomalous behaviour. The visualiser will include graphed and textual data from TCP, ping, traceroute and DNS measurements.
	\paragraph{}
	We evaluated the visualiser's ability and effectiveness in communicating huge amounts of network data both to technical and non-technical users. A user centred design was adopted to design the visualiser and usability tests were conducted to assess the usability and accuracy of the visualiser.
	\paragraph{}
	
	\end{abstract}
	\begin{CCSXML}
		<ccs2012>
		<concept>
		<concept_id>10003033.10003079.10011704</concept_id>
		<concept_desc>Networks~Network measurement</concept_desc>
		<concept_significance>100</concept_significance>
		</concept>
		</ccs2012>
	\end{CCSXML}
	\ccsdesc[100]{Networks~Network measurement}
	\keywords
	Community Networks, Quality of Service
%\	
%\input{sections/background_and_intro}
%\input{sections/introduction}
%input{sections/background}
%\input{sections/literature_review}
%\input{sections/design_and_implementaton}
%\input{sections/final_evaluation_and_results}
%\input{sections/limitations}
%\input{sections/conclusions_and_future_work}
	%\nocite{*}
	\bibliographystyle{acm}
	\bibliography{references}
%	\input{sections/appendix}
\end{document}